\documentclass[12pt]{article}
%\sloppy 
\textheight24.7cm                                                              
\textwidth18.0cm                                                               
\topmargin-1.5cm                                                               
\oddsidemargin-1.0cm                                                            
\evensidemargin0.0cm
\usepackage{times}
\usepackage{amsmath}
%\renewcommand{\baselinestretch}{2}
\newcommand{\be}[1]{\begin{equation} \label{#1}}
\newcommand{\ee}{\end{equation}}
\newcommand{\bea}[1]{\begin{eqnarray} \label{#1}}
\newcommand{\eea}{\end{eqnarray}}
\newcommand{\bean}{\begin{eqnarray*}}
\newcommand{\eean}{\end{eqnarray*}}

\newcommand{\non}{\nonumber\\}
\newcommand{\eq}[1]{(\ref{#1})}
\newcommand{\difp}[2]{\frac{\partial #1}{\partial #2}}
\newcommand{\br}{{\bf r}}
\newcommand{\bR}{{\bf R}}
\newcommand{\bA}{{\bf A}}
\newcommand{\bB}{{\bf B}}
\newcommand{\bE}{{\bf E}}
\newcommand{\bm}{{\bf m}}
\newcommand{\bn}{{\bf n}}
\newcommand{\bN}{{\bf N}}
\newcommand{\bp}{{\bf p}}
\newcommand{\bF}{{\bf F}}
\newcommand{\bz}{{\bf z}}
\newcommand{\bZ}{{\bf Z}}
\newcommand{\bV}{{\bf V}}
\newcommand{\bv}{{\bf v}}
\newcommand{\bu}{{\bf u}}
\newcommand{\bx}{{\bf x}}
\newcommand{\bX}{{\bf X}}
\newcommand{\bJ}{{\bf J}}
\newcommand{\bj}{{\bf j}}
\newcommand{\bk}{{\bf k}}
\newcommand{\bTheta}{{\bf \Theta}}
\newcommand{\btheta}{{\boldsymbol\theta}}
\newcommand{\bOmega}{{\bf \Omega}}
\newcommand{\bomega}{{\boldsymbol\omega}}
\newcommand{\brho}{{\boldsymbol\rho}}
\newcommand{\rd}{{\rm d}}
\newcommand{\rJ}{{\rm J}}
\newcommand{\ph}{{\varphi}}
\newcommand{\te}{\theta}
\newcommand{\tht}{\vartheta}
\newcommand{\vpar}{v_\parallel}
\newcommand{\vparkb}{v_{\parallel k b}}
\newcommand{\vparkm}{v_{\parallel k m}}
\newcommand{\Jpar}{J_\parallel}
\newcommand{\ppar}{p_\parallel}
\newcommand{\Bpstar}{B_\parallel^*}
\newcommand{\intpi}{\int\limits_{0}^{2\pi}}
\newcommand{\summ}{\sum \limits_{m=-\infty}^\infty}
\newcommand{\tb}{\tau_b(\uv)}
\newcommand{\bh}{{\bf h}}
\newcommand{\cE}{{\cal E}}
\newcommand{\cB}{{\cal B}}
\newcommand{\cg}{{\cal G}}
\newcommand{\odtwo}[2]{\frac{\rd #1}{\rd #2}}
\newcommand{\pdone}[1]{\frac{\partial}{\partial #1}}
\newcommand{\pdtwo}[2]{\frac{\partial #1}{\partial #2}}
\newcommand{\ds}{\displaystyle}
%\newcommand{\bc}{\begin{center} 
%\newcommand{\ec}{\end{center}}
%\input{prespan}

\begin{document}

\section{Canonical flux coordinates obtained from VMEC data}

\subsection{VMEC data}
\label{ssec:vmecdata}

VMEC code provides the data in his own flux coordinates $(s,\theta,\varphi)$ where 
$s=\psi_{\rm tor}/\psi_{\rm tor}^{(a)}$ is the toroidal flux $\psi_{\rm tor}$ normalized 
by its value $\psi_{\rm tor}^{(a)}$ at the last closed magnetic surface, 
$\theta$ - poloidal angle of VMEC and $\varphi$ - usual azimuth of cylindrical coordinates.
Angles here do not make a straight field line coordinate system.
Output of the code consists of three 3D function which are poloidal coordinates $R(s,\theta,\varphi)$ 
and $Z(s,\theta,\varphi)$, and the stream function $\lambda(s,\theta,\varphi)$, and of three 1D functions
which are the rotational transform angle $\iota(s)$, normalized poloidal flux $\psi_{\rm pol}(s)$ and
toroidal magnetic field flux $\psi_{\rm tor}(s)$ (linear function). Normalized magnetic field fluxes 
differ from full fluxes $\Psi_{\rm pol}$ and $\Psi_{\rm tor}$ by a factor of $2\pi$,
$\Psi_{\rm pol}=2\pi\psi_{\rm pol}$, $\Psi_{\rm tor}=2\pi\psi_{\rm tor}$.

\noindent
Stream function $\lambda$ determines the conversion to straight field line coordinates $(s,\vartheta,\varphi)$
which are so-called symmetry flux coordinates with poloidal angle
\be{defpolangle}
\vartheta = \theta + \lambda(s,\theta,\varphi).
\ee
There-dimensional functions are represented in the form of stellarator-symmetric Fourier series,
\bea{stelsymseries}
R(s,\theta,\varphi) &=& \sum_{k=1}^{N_F} R_{m_k n_k}(s)\cos(m_k\theta-n_k\varphi),
\nonumber \\
Z(s,\theta,\varphi) &=& \sum_{k=1}^{N_F} Z_{m_k n_k}(s)\sin(m_k\theta-n_k\varphi),
\\
\lambda(s,\theta,\varphi) &=& \sum_{k=1}^{N_F} \lambda_{m_k n_k}(s)\sin(m_k\theta-n_k\varphi),
\nonumber
\eea
where poloidal, $m_k$, and toroidal, $n_k$, Fourier harmonic indices are numbered in the single list by list index $k$
with total number of harmonics $N_F$. Fourier amplitudes $R_{m_k n_k}$, $Z_{m_k n_k}$ and $\lambda_{m_k n_k}$ as well
as 1D quantities $\iota$, $\psi_{\rm pol}$ and $\psi_{\rm tor}$ are given at knots $s=s_j$ of an equidistant grid 
in $s$ with $1\le j \le N_s$ such that $s_{j_{N_s}}=1$.

\noindent
With this, magnetic field is presented in the form
\be{magfieldform}
\bB=\nabla\varphi\times\nabla\psi_{\rm pol}+\nabla\psi_{\rm tor}\times\nabla\left(\theta+\lambda\right)
=\nabla\times\left(\psi_{\rm tor}\nabla\vartheta-\psi_{\rm pol}\nabla\varphi\right)
=\nabla\times\left(A_\vartheta\nabla\vartheta+A_\varphi\nabla\varphi\right),
\ee
where $A_\vartheta=\psi_{\rm tor}$ and $A_\varphi=-\psi_{\rm pol}$ are covariant components of vector-potential.
Since $\iota$, poloidal and toroidal fluxes are linked together by the relation~\eq{iotadef} below, 
information about these
1D quantities is redundant. Therefore, one should use only $\iota(s)$ or $\psi_{\rm pol}(s)$ for data consistency,
and reconstruct the other one from~\eq{iotadef}.

\subsection{Flux coordinates with $B_r=0$}

\noindent
From now on we would formally introduce more general notation for the flux surfaces label $r$, which would 
play a role of radial coordinate, although all results will be obtained later setting $r=s$.
In order to find suitable, ``canonical'' coordinates with zero radial covariant magnetic field component, 
$B_r=0$, we start from some straight field line flux coordinates $\bx=(r,\vartheta,\varphi)$ which in our case
are symmetry flux coordinates. 
General properties of straight field line flux coordinates are the following.
\begin{itemize}
\item 
Vector potential is already of the desired form,
\be{Aflux}
\bA = A_\vartheta\nabla\vartheta+A_\varphi\nabla\varphi,
\ee
where covariant components $A_\vartheta=\psi_{\rm tor}(r)$ and 
$A_\varphi=-\psi_{\rm pol}(r)$ are expressed through the
poloidal and toroidal fluxes divided by $2\pi$, respectively.
\item
Magnetic field, consequently, is expressed through the contra-variant components
as
\be{Bflux}
\bB = 
\nabla A_\vartheta\times\nabla\vartheta
+\nabla A_\varphi\times\nabla\varphi
=
 A_\vartheta^\prime \nabla r\times\nabla\vartheta
+A_\varphi^\prime \nabla r\times\nabla\varphi
=
\frac{A_\vartheta^\prime}{\sqrt{g}}\difp{\br}{\varphi}
-\frac{A_\varphi^\prime}{\sqrt{g}}\difp{\br}{\vartheta}
=
B^k \difp{\br}{x^k},
\ee
where metric determinant $g$ is given by
\be{sqrtg}
\sqrt{g}=\difp{\br}{r}\cdot\difp{\br}{\varphi}\times\difp{\br}{\vartheta}
=\frac{1}{\nabla r \cdot \nabla\vartheta\times\nabla\varphi}
\ee
Namely, Eq.~\eq{Bflux} gives for the contra-variant components
\be{conraBcomp}
B^r=0,  
\qquad B^\vartheta 
= -\frac{A_\varphi^\prime}{\sqrt{g}}
=\frac{\psi_{\rm pol}^\prime}{\sqrt{g}},
\qquad B^\varphi 
= \frac{A_\vartheta^\prime}{\sqrt{g}}
= \frac{\psi_{\rm tor}^\prime}{\sqrt{g}},
\ee
and, as a consequence, 
the property of straight field lines $B^\vartheta=\iota(r) B^\varphi$
where
\be{iotadef}
\iota(r)=\frac{1}{q(r)}=-\frac{A_\vartheta^\prime}{A_\varphi^\prime}
=\frac{\psi_{\rm pol}^\prime(r)}
{\psi_{\rm tor}^\prime(r)}=
\frac{\rd \psi_{\rm pol}}{\rd \psi_{\rm tor}}
\ee
is the rotational transform ($q$ is a safety factor).
\item
Covariant field components follow from the usual tensor transformation
\be{bcovar}
B_i = g_{ij}B^j=g_{i\vartheta}B^\vartheta+g_{i\varphi}B^\varphi
=\difp{\br}{x^i}\cdot\difp{\br}{\vartheta}B^\vartheta+\difp{\br}{x^i}\cdot\difp{\br}{\varphi}B^\varphi.
\ee
\item
Different straight field line flux coordinate systems $\bx=(r,\vartheta,\varphi)$ and 
$\bx_c=(r_c,\vartheta_c,\varphi_c)$ (which always assume 
that $r_c=f(r)$ where $f(r)$ is a monotonous function) 
are linked with each other by the transform of angles
\be{transangles}
\vartheta_c = \vartheta+\iota(r)G(\bx),
\qquad
\varphi_c = \varphi+G(\bx),
\ee
where $G(\bx)$ is a single valued function 
(periodic function of angles $\vartheta$ and $\varphi$).
Since all properties~\eq{Bflux}-\eq{conraBcomp} follow from~\eq{Aflux}
let us check its invariance
\be{Acheck}
\bA=
A_\vartheta\nabla\left(\vartheta+\iota G\right)
+
A_\varphi\nabla\left(\varphi+ G\right)
=
A_\vartheta\nabla\vartheta+A_\varphi\nabla\varphi
+\nabla\left(\iota A_\vartheta G+A_\varphi G\right),
\ee
where we used~\eq{iotadef}. Last term is a gradient and can be removed from
the vector potential (what means a gauge transformation) which leaves then co-variant components unchanged.
\end{itemize}
In the following we will not modify the flux surface label such that $r_c=r$.
When looking for the coordinate system $\bx_c$ with zero co-variant radial magnetic 
field component it is more convenient to look for the inverse
transform
\be{invtrans}
\vartheta = \vartheta_c+\iota(r)G_c(\bx_c),
\qquad
\varphi = \varphi_c+G_c(\bx_c).
\ee
From the transformation of co-variant components of the magnetic field,
\bea{Bcovar}
B_k^c=B_j\difp{x^j}{x_c^k},
\eea
and the condition $B^c_r=0$ we obtain the nonlinear ODE for $G_c$,
\be{eqforG}
\difp{}{r}G_c(r,\vartheta_c,\varphi_c)=
-\frac{B_r(r,\vartheta,\varphi)
+\iota^\prime(r) B_\vartheta(r,\vartheta,\varphi) G_c(r,\vartheta_c,\varphi_c)}
{\iota(r)B_\vartheta(r,\vartheta,\varphi)
+B_\varphi(r,\vartheta,\varphi)},
\ee
where $\vartheta=\vartheta(\bx_c)$ and $\varphi=\varphi(\bx_c)$ must be substituted from Eq.~\eq{invtrans}.
In the tokamak case, magnetic field components are independent of the 
toroidal angle and, therefore, $G_c=G_c(r,\vartheta_c)$. As a result, components
of the magnetic field are independent of $\varphi_c$ in the new coordinates
too.

\noindent
Set of quantities required for the formulation of guiding center equations in the canonical form
is covariant components of vector potential, $A_\vartheta^c=A_\vartheta$ and $A_\varphi^c=A_\varphi$,
magnetic field strength $B$ and covariant components of the magnetic field $B_\vartheta^c$ and $B_\varphi^c$.
Using the relations~\eq{Bcovar} and~\eq{invtrans},
the latter two are expressed with help of function $G_c$ via corresponding components 
$B_\vartheta$ and $B_\varphi$ in symmetry flux coordinates as follows,
\bea{covarBcan}
B_\vartheta^c
&=&
B_\vartheta + \left(\iota B_\vartheta+B_\varphi\right)\difp{}{\vartheta_c}G_c(\bx_c),
\nonumber \\
B_\varphi^c
&=&
B_\varphi + \left(\iota B_\vartheta+B_\varphi\right)\difp{}{\varphi_c}G_c(\bx_c).
\eea

\subsection{Direct transformation from VMEC coordinates to canonical coordinates}

\noindent
As one can see from the structure of VMEC data in section~\ref{ssec:vmecdata}
3D functions are explicitly given as functions of intrinsic VMEC coordinates $\bx_V=(s,\theta,\varphi)$
but of symmetry flux coordinates $\bx=(s,\vartheta,\varphi)$. It is convenient to avoid transformation
of 3D functions to functions of $\bx$ and obtain final results in canonical coordinates $\bx_c$
directly from VMEC coordinates $\bx_V$. For this purpose we denote covariant magnetic filed 
components in symmetry flux coordinates $B_k$ expressed as functions of VMEC coordinates $\bx_V$ as
follows
\be{dentcov}
B_k\left(r,\vartheta(r,\theta,\varphi),\varphi\right)=\cB_k(r,\theta,\varphi).
\ee
Then, equation~\eq{eqforG} for transformation function $G_c$ becomes
\be{eqforG_VMEC}
\difp{}{r}G_c(\bx_c)\equiv
\difp{}{r}G_c(r,\vartheta_c,\varphi_c)=
-\frac{\cB_r\left(r,\theta(\bx_c),\varphi(\bx_c)\right)
+\iota^\prime(r) \cB_\vartheta\left(r,\theta(\bx_c),\varphi\right) G_c(r,\vartheta_c,\varphi_c)}
{\iota(r)\cB_\vartheta\left(r,\theta(\bx_c),\varphi(\bx_c)\right)
+\cB_\varphi\left(r,\theta(\bx_c),\varphi(\bx_c)\right)},
\ee
where functions $\theta(\bx_c)$ and $\varphi(\bx_c)$ are implicitly defined by
\be{vmec_via_can}
\theta(\bx_c) + \lambda\left(r,\theta(\bx_c),\varphi(\bx_c)\right) = \vartheta_c+\iota(r)G_c(\bx_c),
\qquad
\varphi(\bx_c) = \varphi_c+G_c(\bx_c).
\ee
which follows from~\eq{invtrans} and~\eq{defpolangle}. First of these functions, $\theta(\bx_c)$, can be
computed by Newton method. The covariant canonical components of the magnetic field~\eq{covarBcan} then are
\bea{covarBcan_VMEC}
B_\vartheta^c(\bx_c)
&=&
\cB_\vartheta\left(r,\theta(\bx_c),\varphi(\bx_c)\right) + 
\left(\iota(r) \cB_\vartheta\left(r,\theta(\bx_c),\varphi(\bx_c)\right)
+\cB_\varphi\left(r,\theta(\bx_c),\varphi(\bx_c)\right)\right)\difp{}{\vartheta_c}G_c(\bx_c),
\nonumber \\
B_\varphi^c(\bx_c)
&=&
\cB_\varphi\left(r,\theta(\bx_c),\varphi(\bx_c)\right) 
+ \left(\iota(r) \cB_\vartheta\left(r,\theta(\bx_c),\varphi(\bx_c)\right)
+\cB_\varphi\left(r,\theta(\bx_c),\varphi(\bx_c)\right)\right)\difp{}{\varphi_c}G_c(\bx_c).
\eea
Since expression~\eq{conraBcomp} for contra-variant components of are the same in all straight field line
flux coordinates and covariant components of vector potential are the same too,
what remains is to obtain metric determinant of canonical coordinates $g_c$ in order to define those components as
\be{conraBcomp_can}
B^r_c(\bx_c)=0,
\qquad B^\vartheta_c(\bx_c)
= -\frac{A_\varphi^\prime(r)}{\sqrt{g_c(\bx_c)}},
\qquad B^\varphi_c(\bx_c)
= \frac{A_\vartheta^\prime(r)}{\sqrt{g_c(\bx_c)}}.
\ee
First, let us compute using Jacobians metric determinant of VMEC coordinates, $g_V$, 
via metric determinant of cylindrical coordinates $(R,\varphi,Z)$ which are known functions of VMEC coordinates 
$\bx_V=(s,\theta,\varphi)=(r,\theta,\varphi)$,
\be{g_VMEC}
\sqrt{g_V(\bx_V)}=\difp{(\br)}{(\bx_V)}=R\difp{(R,\varphi,Z)}{(r,\theta,\varphi)}=R\difp{(R,Z)}{(\theta,r)}
=R\left(\difp{R}{\theta}\difp{Z}{r}-\difp{Z}{\theta}\difp{R}{r}\right).
\ee
Metric determinant of canonical coordinates, $g_c$, follows then as
\bea{g_can}
\sqrt{g_c(\bx_c)}
&=&
\difp{(\br)}{(\bx_c)}=\difp{(\br)}{(\bx_V)}\difp{(\bx_V)}{(\bx_c)}
=\sqrt{g_V(\bx_V)}\difp{(r,\theta,\varphi)}{(r,\theta_c,\varphi_c)}
\nonumber \\
&=&
\sqrt{g_V(\bx_V(\bx_c))}\left(\difp{\theta(\bx_c)}{\vartheta_c}\difp{\varphi(\bx_c)}{\varphi_c}
-\difp{\varphi(\bx_c)}{\vartheta_c}\difp{\theta(\bx_c)}{\varphi_c}\right)
\nonumber \\
&=&
\sqrt{\cg\left(\bx_V(\bx_c)\right)}\left(1+\iota(r) \difp{G_c(\bx_c)}{\vartheta_c}+\difp{G_c(\bx_c)}{\varphi_c}\right),
\eea
where dependencies $\bx_V(\bx_c)$ of VMEC coordinates on canonical coordinates are known via~\eq{vmec_via_can},
and with $\sqrt{\cg(\bx_V)}$ we denote the Jacobian of symmetry flux coordinates $\sqrt{g}$ as function of VMEC coordinates,
$\sqrt{\cg(\bx_V)}=\sqrt{g\left(\bx(\bx_V)\right)}$.

\noindent
What is still missing here, it is functions $\cB_k$ used in Eq.~\eq{eqforG_VMEC} and in~\eq{covarBcan_VMEC}. 
We obtain these functions from contra-variant components in symmetry flux coordinates, $B^k$
\be{covar_symm}
B_k = g_{k\vartheta} B^\vartheta+g_{k\varphi} B^\varphi
=\frac{A^\prime_\vartheta}{\sqrt{g}}\left(g_{k\varphi}+\iota g_{k\vartheta}\right),
\ee
where we used~\eq{conraBcomp} and~\eq{iotadef}. To make~\eq{covar_symm} a definition of functions $\cB_k$
we need to obtain $\sqrt{g}$ and $g_{ik}$ of symmetry flux coordinates as functions of VMEC coordinates.
First of these follows as
\be{sqg_sym}
\sqrt{g}=\sqrt{g_V}\difp{(r,\theta,\varphi)}{(r,\vartheta,\varphi)}
= \sqrt{g_V}\left(\difp{\vartheta}{\theta}\right)^{-1}
= \sqrt{g_V(\bx_V)}\left(1+\difp{}{\theta}\lambda(\bx_V)\right)^{-1}.
\ee
Metric tensor of symmetry flux coordinates $g_{ik}$ is obtained from metric tensor of VMEC coordinates $g_{ik}^V$
which has the following components
\bea{gik_VMEC}
g^V_{rr} &=& \left(\difp{R}{r}\right)^2+\left(\difp{Z}{r}\right)^2,
\nonumber \\
g^V_{r\theta} &=& g^V_{\theta r} = \difp{R}{r}\difp{R}{\theta}+\difp{Z}{r}\difp{Z}{\theta},
\nonumber \\
g^V_{r\varphi} &=& g^V_{\varphi r} = \difp{R}{r}\difp{R}{\varphi}+\difp{Z}{r}\difp{Z}{\varphi},
\nonumber \\
g^V_{\theta\theta} &=& \left(\difp{R}{\theta}\right)^2+\left(\difp{Z}{\theta}\right)^2,
\nonumber \\
g^V_{\theta\varphi} &=& g^V_{\varphi \theta} = \difp{R}{\theta}\difp{R}{\varphi}+\difp{Z}{\theta}\difp{Z}{\varphi},
\nonumber \\
g^V_{\varphi\varphi} &=& R^2+\left(\difp{R}{\varphi}\right)^2+\left(\difp{Z}{\varphi}\right)^2.
\eea
In order to use
\be{transV-c}
g_{ik}=g_{mn}^V c^m_i c^n_k, \qquad c^m_i \equiv \difp{x_V^m}{x^i}
\ee
we calculate the inverse matrix $\bar c^i_k$ such that
\be{invmatc}
\bar c^i_k c^k_j = \delta^i_j, \qquad \bar c^i_j = \difp{x^i}{x_V^j}.
\ee
Explicitly, components of this matrix are
\bea{cbar_elements}
& & \bar c^r_r=1, \qquad \bar c^r_\theta=0, \qquad \bar c^r_\varphi=0,
\nonumber \\
& & \bar c^\vartheta_r=\difp{\lambda}{r}, 
\qquad \bar c^\vartheta_\theta=1+\difp{\lambda}{\theta}, \qquad \bar c^\vartheta_\varphi=\difp{\lambda}{\varphi},
\nonumber \\
& & \bar c^\varphi_r=0, \qquad \bar c^\varphi_\theta=0, \qquad \bar c^\varphi_\varphi=1.
\eea
Respectively, components of matrix $c^i_j$ are
\bea{c_elements}
& & c^r_r=1, \qquad c^r_\vartheta=0, \qquad c^r_\varphi=0,
\nonumber \\
& & c^\theta_r=-\difp{\lambda}{r}\left(1+\difp{\lambda}{\theta}\right)^{-1}, 
\qquad c^\theta_\vartheta=\left(1+\difp{\lambda}{\theta}\right)^{-1}, 
\qquad c^\theta_\varphi=-\difp{\lambda}{\varphi}\left(1+\difp{\lambda}{\theta}\right)^{-1},
\nonumber \\
& & c^\varphi_r=0, \qquad c^\varphi_\vartheta=0, \qquad c^\varphi_\varphi=1.
\eea

\subsection{Numerical realization}

\noindent
Equation~\eq{eqforG_VMEC} is a nonlinear ODE with respect to variable $r$ where variables $\vartheta_c$
and $\varphi_c$ play a role of parameters. Therefore, family of solutions to this equation non-trivially
depends on initial radial integration point and a starting value of $G_c$ which can be an arbitrary function
of parameters $\vartheta_c$ and $\varphi_c$. Our choice of initial integration point is half radius because
near the last closed magnetic surface and near the magnetic axis VMEC data suffers from numerical pollution.
Integration constant will be fixed to zero.

For numerics, we spline $R(\bx_V)$, $Z(\bx_V)$ and $\lambda(\bx_V)$ in 3D using 5-th order splines (periodic
in $\theta$ and $\varphi$) and spline $A_\varphi$ and $A_\vartheta$ in 1D. Then all necessary derivatives of 
these functions are available. Equation~\eq{eqforG_VMEC} is solved then by RK4/5 for each node of the 
equidistant grid in $\vartheta_c$ and $\varphi_c$, and the transformation function $G_c$ is again splined
in 3D on $\bx_c$ grid. Using these two kinds of 3D splines (on $\bx_V$ grid and on $\bx_c$ grid) the covariant
components $B^c_\vartheta$ and $B^c_\varphi$ and metric determinant $\sqrt{g_c}$ can be computed on $\bx_c$
grid. The rest quantities (contra-variant components $B_c^\vartheta$ and $B_c^\varphi$, module of magnetic field 
$B$) can be computed using 3D spline interpolation of those 3 quantities and 1D spline interpolation for 
covariant components of vector potentials $A_\vartheta$ and $A_\varphi$.



\subsection{Canonical form of guiding center equations} 

Guiding center Lagrangian, which has in general curvilinear coordinates $\bx$ the following form,
\be{lagrgencoord}
L_{gc} = \left(mv_\parallel h_k+\frac{e}{c}A_k\right)\dot x^k-J_\perp\dot\phi-H,
\ee
where $h_k = B_k /B$, and the Hamiltonian is
\be{gcham}
H=J_\perp\omega_c+\frac{mv_\parallel^2}{2}+e\Phi,
\ee
in canonical straight field line coordinates $\bx_c$ where covariant radial components of both, vector potential
and magnetic field are zeros, $A_r^c=B_r^c=0$, takes a short form
\be{lagrcancoord}
L_{gc} = 
\left(mv_\parallel h_\vartheta^c+\frac{e}{c}A_\vartheta^c\right)\dot \vartheta_c
+
\left(mv_\parallel h_\varphi^c+\frac{e}{c}A_\varphi^c\right)\dot \varphi_c
-
J_\perp\dot\phi-H.
\ee
\noindent
Introducing in~\eq{lagrcancoord} the notation
\be{canmomenta}
p_\vartheta=mv_\parallel h_\vartheta^c+\frac{e}{c}A_\vartheta^c,
\qquad
p_\varphi=mv_\parallel h_\varphi^c+\frac{e}{c}A_\varphi^c,
\ee
we obtain the canonical form of the Lagrangian,
\be{Lagr_can}
L_{gc}=p_\vartheta\dot\vartheta_c+p_\varphi\dot\varphi_c- J_\perp\dot\phi-H.
\ee
Note that we did not specify here the independent variables. Lagrangian does 
not depend on this choice. If we choose $p_\vartheta$ and $p_\varphi$ instead
of $r$ and $H$, we obtain Hamiltonian equations of motion,
\be{hameqs}
\dot J_\perp = 0, \quad \dot\phi=-\difp{H}{J_\perp},
\qquad
\dot p_\vartheta = - \difp{H}{\vartheta_c}, \quad 
\dot \vartheta_c = \difp{H}{p_\vartheta},
\qquad
\dot p_\varphi = - \difp{H}{\varphi_c}, \quad 
\dot \varphi_c = \difp{H}{p_\varphi},
\ee
where $H=H(\vartheta_c,\varphi_c,J_\perp,p_\vartheta,p_\varphi)$. These equations
are still different from the usual form because of the sign of the Hamiltonian
in the first pair (sign in the equation for $\dot\phi$). Standard form can be 
obtained by either changing the sign of $J_\perp\rightarrow -J_\perp$ 
(usual way in the literature) or by changing the sign of the gyrophase,
$\phi\rightarrow-\phi$, as in our cylindrical model where the gyrophase is
counted clockwise (in mathematically negative direction).

\section{Boozer flux coordinates obtained from VMEC data}

\subsection{Generating function from transformation of covariant field components}
\label{ssec:covartrans}

We denote here a set of Boozer flux coordinates as $\bx_B=(r,\vartheta_B,\varphi_B)$.
In these coordinates, field is fully described by two 3D functions, namely, module
of $B=B(r,\vartheta_B,\varphi_B)$ and covariant radial field components 
$B_r^B=B_r^B(r,\vartheta_B,\varphi_B)$ while the other two covariant components
are 1D, $B_\vartheta^B=B_\vartheta^B(r)$ and $B_\varphi^B=B_\varphi^B(r)$.
Our task is to obtain generating function $G_B(\bx_b)$ in the inverse transform
similar to~\eq{invtrans}. Given are field components $B_k$ in symmetry flux coordinates
$\bx=(r,\vartheta,\varphi)$ as functions of VMEC coordinates $\bx_V=(r,\theta,\varphi)$ and
stream function $\lambda(\bx_V)$ linking $\theta$ and $\vartheta$ via~\eq{defpolangle}.
With this, we have covariant field components in VMEC coordinates given as
\be{covvmec}
B_\theta^V=\left(1+\difp{\lambda}{\theta}\right)B_\vartheta,
\qquad
B_\varphi^V=B_\varphi+\difp{\lambda}{\varphi}B_\vartheta.
\ee
Using Stoke's theorem,
\be{stokes}
\frac{4\pi}{c}\int\limits_S \rd{\bf S}\cdot\bj=\oint\limits_{\partial S}\rd {\bf l}\cdot\bB,
\ee
where $\rd{\bf S}$ is the element of area $S$ bounded by the line $\partial S$, and $\rd {\bf l}$
is the element of this bounding line. Choosing for the bounding line a poloidal or toroidal contour
whose elements can be described in any flux coordinate system, including VMEC coordinates system $\bx_V$,
as
\be{lineels}
\rd {\bf l}_{\rm pol}=\difp{\br(\bx_V)}{\theta}\rd\theta,
\qquad
\rd {\bf l}_{\rm tor}=\difp{\br(\bx_V)}{\varphi}\rd\varphi,
\ee
so that both line elements are tangential to the magnetic flux surface $r=const$ we obtain in the 
l.h.s of~\eq{stokes} toroidal and poloidal currents through the areas $S_{\rm pol}$ and $S_{\rm tor}$ 
limited by poloidal and toroidal contours $\partial S_{\rm pol}$ and $\partial S_{\rm tor}$,
respectively,
\be{currents}
\int\limits_{S_{\rm pol}} \rd{\bf S}\cdot\bj=I_{\rm tor}(r),
\qquad
\int\limits_{S_{\rm tor}} \rd{\bf S}\cdot\bj=I_{\rm pol}(r).
\ee
Thus
\bea{stokespolotor}
\frac{4\pi}{c}I_{\rm tor}(r)
&=&
\oint\limits_{\partial S_{\rm pol}}\rd {\bf l}_{\rm pol}\cdot\bB
=\int\limits_0^{2\pi}\rd\theta \difp{\br(\bx_V)}{\theta}\cdot \bB
=\int\limits_0^{2\pi}\rd\theta B_\theta^V(\bx_V),
\nonumber \\
\frac{4\pi}{c}I_{\rm pol}(r)
&=&
\oint\limits_{\partial S_{\rm tor}}\rd {\bf l}_{\rm tor}\cdot\bB
=\int\limits_0^{2\pi}\rd\varphi \difp{\br(\bx_V)}{\varphi}\cdot \bB
=\int\limits_0^{2\pi}\rd\varphi B_\varphi^V(\bx_V).
\eea
Since Boozer components are constant on flux surface, we obtain
\be{boozcur}
\frac{2}{c}I_{\rm tor}(r)=B_\vartheta^B(r),
\qquad
\frac{2}{c}I_{\rm pol}(r)=B_\varphi^B(r),
\ee
and, respectively,
\be{boozcovar}
B_\vartheta^B(r)=\frac{1}{2\pi}\int\limits_0^{2\pi}\rd\theta B_\theta^V(\bx_V),
\qquad
B_\varphi^B(r)=\frac{1}{2\pi}\int\limits_0^{2\pi}\rd\varphi B_\varphi^V(\bx_V).
\ee
For a perfect equilibrium, integrals~\eq{boozcovar} are independent of $\varphi$
and $\theta$, respectively, however some weak dependence may appear due to numerical
imperfections. Therefore, we compute Boozer components as double integrals
\be{boozcovar_double}
B_{\vartheta,\varphi}^B(r)=\frac{1}{4\pi^2}
\int\limits_0^{2\pi}\rd\theta 
\int\limits_0^{2\pi}\rd\varphi 
B_{\theta,\varphi}^V(\bx_V),
\ee
well approximated by the sums over equidistant grid
\be{boozcovar_sum}
B_{\vartheta,\varphi}^B(r)\approx\frac{1}{N_\theta N_\varphi}
\sum\limits_{i_\theta=1}^{N_\theta} \sum\limits_{i_\varphi=1}^{N_\varphi}
B_{\theta,\varphi}^V\left(r,\frac{2 \pi i_\theta}{N_\theta},\frac{2\pi i_\varphi}{N_\varphi}\right).
\ee
Since we get Boozer's covariant angular components in this easy way, we can find an equation for the
forward transformation function $G_F(\bx_V)=G_B(\bx_B(\bx_V))$ from the transform of covariant components
to symmetry flux variables where
\be{symmtobooz}
\vartheta_B=\vartheta+\iota G_F, \qquad \varphi_B=\varphi+G_F.
\ee
Namely, we get
\be{covtrans}
B_\vartheta=B_\vartheta^B+
\left(\iota B_\vartheta^B+B_\varphi^B\right)\difp{(G_F,\varphi)}{(\vartheta,\varphi)},
\qquad
B_\varphi=B_\varphi^B+\left(\iota B_\vartheta^B+B_\varphi^B\right)\difp{(\vartheta,G_F)}{(\vartheta,\varphi)},
\ee
where we used Jacobians to indicate fixed variable during differentiation.
Using~\eq{defpolangle} and properties of the Jacobians for transformation to derivatives over VMEC variables
we finally obtain
\bea{dGdphi}
\difp{G_F}{\varphi}\equiv \difp{(\theta,G_F)}{(\theta,\varphi)}
&=&
\frac{1}{\iota B_\vartheta^B+B_\varphi^B}\left(B_\varphi-B_\varphi^B 
+\difp{\lambda}{\varphi}\left(B_\vartheta-B_\vartheta^B\right)
\right)
\nonumber \\
&=&
\frac{1}{\iota B_\vartheta^B+B_\varphi^B}\left(B_\varphi^V-B_\varphi^B 
-\difp{\lambda}{\varphi}B_\vartheta^B\right)
\eea
with~\eq{covvmec} used in the last expression. Similarly we obtain another partial derivative,
\be{dGdtheta}
\difp{G_F}{\theta}\equiv \difp{(G_F,\varphi)}{(\theta,\varphi)}
=
\frac{1}{\iota B_\vartheta^B+B_\varphi^B}\left(B_\theta^V
-B_\vartheta^B\left(1+\difp{\lambda}{\theta}\right)\right).
\ee
Integrating~\eq{dGdtheta} for $\varphi=0$ we obtain $G_F(r,\theta,0)$ which is a starting value
for integration of~\eq{dGdphi} giving $G_F(\bx_V)$ up to a flux surface constant $G_F(r,0,0)$,
\bea{G_F}
G_F(r,\theta,\varphi)
&=&
G_F(r,0,0)+
\frac{1}{\iota B_\vartheta^B(r)+B_\varphi^B(r)}
\left(
\int\limits_0^\varphi\rd\varphi^\prime B_\varphi^V(r,\theta,\varphi^\prime)
-\frac{\varphi}{2\pi}\int\limits_0^{2\pi}\rd\varphi^\prime B_\varphi^V(r,\theta,\varphi^\prime)
\right.
\nonumber \\
&+&
\left.
\int\limits_0^\theta\rd\theta^\prime B_\theta^V(r,\theta^\prime,0)
-\frac{\theta}{2\pi}\int\limits_0^{2\pi}\rd\theta^\prime B_\theta^V(r,\theta^\prime,0)
-B_\vartheta^B(r)\left(\lambda(r,\theta,\varphi)-\lambda(r,0,0)\right)
\right).
\eea
In this form where we used Eqs.~\eq{boozcovar} for some occurrences of covariant Boozer components,
generating function is perfectly periodic even in presence of numerical imperfections of the equilibrium.
Integrals in~\eq{G_F} are computed uploading field components to the equidistant grid in $(\theta,\varphi)$
and using periodic spline approximation.

With this, we obtain two transformation functions specified on the equidistant $(\theta,\varphi)$ grid,
\be{deltp_BV}
\Delta\vartheta_{BV}(r,\theta,\varphi)\equiv\vartheta_B-\theta
=\iota(r) G_F(r,\theta,\varphi)+\lambda(r,\theta,\varphi),
\qquad
\Delta\varphi_{BV}(r,\theta,\varphi)\equiv \varphi_B-\varphi
=G_F(r,\theta,\varphi).
\ee
These function map the equidistant grid in $(\theta,\varphi)$ into non-equidistant grid in
$(\vartheta_B,\varphi_B)$. In order to find the inverse transform $\theta=\theta(r,\vartheta_B,\varphi_B)$
and $\varphi=\varphi(r,\vartheta_B,\varphi_B)$ we simply re-interpolate functions $\Delta\vartheta_{BV}$
and $\Delta\varphi_{BV}$ to the new, equidistant grid in $(\vartheta_B,\varphi_B)$ using Lagrange polynomial
interpolation on the old non-equidistant $(\vartheta_B,\varphi_B)$ grid. This is performed subsequently
re-interpolating over $\vartheta_B$ keeping $\varphi$ fixed (one needs previously to extend the range of $\theta$
variable from $[0,2\pi]$ to $[-2\pi,4\pi]$ in order to avoid problems with periodic boundaries)
to get $\Delta\vartheta_{BV}(r,\theta(\vartheta_B,\varphi),\varphi)$ and
$\Delta\varphi_{BV}(r,\theta(\vartheta_B,\varphi),\varphi)$
and then over $\varphi_B$ keeping $\vartheta_B$ fixed (similar range extension is needed than for the toroidal 
angle).
Re-interpolating $G_F(r,\theta,\varphi)$ and $B(r,\theta,\varphi)$ together with transformation functions
one obtains $G_B(\bx_B)$ and $B(\bx_B)$ specified on the equidistant grid of Boozer angles.
Note that all the functions which are re-interpolated are periodic functions of angles - no problems appear
due to multi-valuedness of angles.

\subsection{Radial component}

For the radial component, we have an expression via symmetry flux components
\bea{B_rsf}
B_r^B &=& 
B_r+B_\vartheta\difp{}{r}\vartheta(\bx_B)+B_\varphi\difp{}{r}\varphi(\bx_B)
\nonumber \\
&=&
B_r-B_\vartheta\difp{}{r}\iota(r)G_B(\bx_B)-B_\varphi\difp{}{r}G_B(\bx_B).
\eea
Re-interpolating covariant symmetry flux components $B_k(\bx_V)$ to the equidistant grid of Boozer angles 
(within the transformation of the previous section) gives us $B_k(\bx_B)$. 
Thus, all quantities in the r.h.s. of~\eq{B_rsf} are specified on the equidistant grid of Boozer angles.
What remains is to compute
radial derivatives by finite difference scheme (we use 4-th order Lagrange polynomial for this).

\bigskip
Finally, we store and spline interpolate up to six 3D functions. Arguments of these functions
indicate equidistant angular grid in particular variables ($\bx_V$ means VMEC grid and $\bx_B$ means Boozer grid).
These functions are $B(\bx_B)$ which is mandatory, $B_r(\bx_B)$ which is optional (needed if one uses usual 
integrator for exact orbits), $\Delta \vartheta_{BV}(\bx_V)$ and $\Delta\varphi_{BV}(\bx_V)$ which are also
mandatory but are needed only for the conversion of initial starting points from VMEC to Boozer coordinates
and backward conversion of final points by 2D Newton method (i.e. they are not needed by the integrators), and 
$\Delta \vartheta_{BV}(\bx_B)$ and $\Delta\varphi_{BV}(\bx_B)$ (optional). Last two quantities
provide the first iteration for the inverse transformation $\bx_B\rightarrow\bx_V$ however they can be safely
ignored since they save only one iteration (normally 4 Newton iterations converge up to computer accuracy).

\end{document}
